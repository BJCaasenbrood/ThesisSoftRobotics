%!TEX root = C:\Users\s118759\Documents\GitHub\ThesisSoftRobotics\main.tex
\section{Configuration spaces}
In contrast to a rigid robot, whose mechanical structure consists of static links and joints, a soft robot lacks the physical notion of joints and therefore cannot be viewed as a single- or multi-body. From a mechanical perspective, a soft robotic system is more closely related to a continuous deformable medium with infinite degrees-of-freedom rather than a hyper-redundant system. Given this notion, the hyper-flexible soft robot is modeled as a one-dimensional Cosserat beam together with the geometrically exact beam theories proposed by \cite{Simo1986}. The main idea is to regard the material solid as a series of infinitesimally thin (semi)-rigid body that co-align with a spatial curve passing through the geometrical center of each cross-section. To characterized the spatial dimension along this curve, let us introduce a spatial parameter $\sigma \in \mathcal{X}$ with $\mathcal{X} \in [0,L] \subset \R$ (where $L \in \Rsp$ is the undeformed length of the soft robot). To further elaborate its physical notation, the spatial parameter $\sigma$ represents a material point inside the hyper-flexible body of soft robot, or more precisely, located at the center of the cross section of the continuous elastic body at $\sigma$. 