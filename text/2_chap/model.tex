%!TEX root = C:\Users\s118759\Documents\GitHub\ThesisSoftRobotics\main.tex
\chapter{Dynamics of Soft Robots}
\label{ch:2-dynamics} 

%!TEX root = /home/brandon/Documents/phd/thesisSoftRobotics/main.tex
\chapter{Fundamentals on \\ \vspace{5mm} Lie Group Theory}
In this chapter, we will discuss the fundamentals on Lie groups and their associated Lie algebras.

\section{Lie group}
A Lie group encompasses the concepts of `group' and `smooth manifold' in a unique embodiment. To be more specific, the Lie Group $\G$ is a smooth manifold whose elements satisfy the group axioms:
\begin{enumerate}
\item Closure: if $g_1,g_2 \in \G$, then $g_1g_2$ is also an element of $\G$,
\item Identity: there exists an element $e$ such that $g e = e g = g$ for any $g \in G$,
\item Inversion: For any $g \in \G$, there exists an element $g^{-1} \in \G$ such that $g g^{-1} = g^{-1}g = e$,
\item Associativity: $(g_1 g_2) g_3 = g_1 (g_2 g_3)$ for any $g_1,g_2,g_3 \in \G$.
\end{enumerate}

The smoothness of the Lie groups intuitively suggests the existence of useful differential geometries. For any elements $g$ on the smooth manifold $\G$, there exists a tangent linear space denoted by $T_{g} \mathcal{G}$. The tangent space of the Lie group at the identity element $e$ is referred to as the associative Lie algebra $\g$ of the group. it allows us to perform algebra computation concerning the Lie group.

\section{Adjoint action on $\SE{3}$}
%!TEX root = C:\Users\s118759\Documents\GitHub\ThesisSoftRobotics\main.tex
\section{Configuration spaces}
In contrast to a rigid robot, whose mechanical structure consists of static links and joints, a soft robot lacks the physical notion of joints and therefore cannot be viewed as a single- or multi-body. From a mechanical perspective, a soft robotic system is more closely related to a continuous deformable medium with infinite degrees-of-freedom rather than a hyper-redundant system. Given this notion, the hyper-flexible soft robot is modeled as a one-dimensional Cosserat beam together with the geometrically exact beam theories proposed by \cite{Simo1986}. The main idea is to regard the material solid as a series of infinitesimally thin (semi)-rigid body that co-align with a spatial curve passing through the geometrical center of each cross-section. To characterized the spatial dimension along this curve, let us introduce a spatial parameter $\sigma \in \mathcal{X}$ with $\mathcal{X} \in [0,L] \subset \R$ (where $L \in \Rsp$ is the undeformed length of the soft robot). To further elaborate its physical notation, the spatial parameter $\sigma$ represents a material point inside the hyper-flexible body of soft robot, or more precisely, located at the center of the cross section of the continuous elastic body at $\sigma$. 
%%!TEX root = /home/brandon/Documents/phd/thesisSoftRobotics/main.tex
\chapter{Fundamentals on \\ \vspace{5mm} Lie Group Theory}
In this chapter, we will discuss the fundamentals on Lie groups and their associated Lie algebras.

\section{Lie group}
A Lie group encompasses the concepts of `group' and `smooth manifold' in a unique embodiment. To be more specific, the Lie Group $\G$ is a smooth manifold whose elements satisfy the group axioms:
\begin{enumerate}
\item Closure: if $g_1,g_2 \in \G$, then $g_1g_2$ is also an element of $\G$,
\item Identity: there exists an element $e$ such that $g e = e g = g$ for any $g \in G$,
\item Inversion: For any $g \in \G$, there exists an element $g^{-1} \in \G$ such that $g g^{-1} = g^{-1}g = e$,
\item Associativity: $(g_1 g_2) g_3 = g_1 (g_2 g_3)$ for any $g_1,g_2,g_3 \in \G$.
\end{enumerate}

The smoothness of the Lie groups intuitively suggests the existence of useful differential geometries. For any elements $g$ on the smooth manifold $\G$, there exists a tangent linear space denoted by $T_{g} \mathcal{G}$. The tangent space of the Lie group at the identity element $e$ is referred to as the associative Lie algebra $\g$ of the group. it allows us to perform algebra computation concerning the Lie group.

\section{Adjoint action on $\SE{3}$}
\newpage
%!TEX root = C:\Users\s118759\Documents\GitHub\ThesisSoftRobotics\main.tex
\newpage
\section{Dynamics through Hamilton's variational principle}
In this section, we derive the dynamical model of the soft robot through Hamilton's variational principle. The principle states that the time evolution of a state vector $q(t)$ between two state instances $q_1 = q(t_1)$ and $q_2 = q(t_2)$ over a fixed time interval $[t_1,t_2]$ is a stationary point regarding an action functional, $\mathcal{S} = \int_{t_1}^{t_2} \mathcal{L}(\vec{q},\dot{q},t) \; dt$ in which $\mathcal{L}(\vec{q},\dot{\vec{q}}) := \mathcal{T}(\vec{q},\dot{\vec{q}}) - \mathcal{V}(\vec{q})$ is the Lagrangian. The extension of Hamilton's principle \cite{Boyer2010} also considers external potential contributions, and can be formally written as
\begin{equation}
\delta(\mathcal{S}) = \int_{t_0}^{t_1} \left[\delta(\mathcal{T}) - \delta(\mathcal{V}) + \delta(\mathcal{W}_{ex}) \right]\; dt = 0,
\end{equation}

\noindent where the operator $\delta(\cdot)$ denotes a variation applied along the trajectory of the system that are fixed at the boundaries of $[t_0,t_1]$, and $\mathcal{W}_{ex}$ is the external virtual work produced by nonconservative external forces to the dynamical system. First, the kinetic energy of the soft robot is defined as 
\begin{equation}
\mathcal{T} := \frac{1}{2}\int_{\mathcal{X}} \vec{\eta}^\tr \mathcal{M} \vec{\eta}\;d\sigma, \label{eq:c2_Tfunc} 
\end{equation}
\noindent where $\mathcal{M} \in \R^{6\times6}$ is a the inertia tensors whose components correspond to the mass and moment of inertias. Then, the variation of the kinetic energy function is given by
\begin{align}
\delta(\mathcal{T}) & = \left. \frac{\p }{\p a} \mathcal{T}(\vec{\eta} + a \delta(\vec{\eta})) \right|_{a = 0}, \notag \\
& = \frac{1}{2} \int_{\mathcal{X}} \delta(\vec{\eta})^\tr \mathcal{M} \vec{\eta} + \vec{\eta}^\tr \mathcal{M} \delta(\vec{\eta}) \; d\sigma, \notag \\
& = \int_{\mathcal{X}} \delta(\vec{\eta})^\tr \mathcal{M} \vec{\eta} \; d\sigma, \label{eq:c2_varham_T}
\end{align}

\noindent Recalling the commutativity of the Lie algebra, we can express the variation as $\delta(\vec{\eta}) = \dot{\vec{\epsilon}} + \ad_{\vec{\eta}} \vec{\epsilon}$. Therefore, substitution into \eqref{eq:c2_varham_T} and followed by integration by parts leads to 
\begin{align}
\int_{t_1}^{t_2} \delta(\mathcal{T}) \; dt& = 
\left[ \int_\mathcal{X} \vec{\epsilon}^\tr \mat{M} \vec{\eta} \right]_{t_0}^{t_1} +  \int_{t_1}^{t_2} \!\! \int_\mathcal{X}  \vec{\epsilon}^\tr \! \left( M\dot{\vec{\eta}} - \ad_{\vec{\eta}}^\tr \! M \vec{\eta}\right)\; d\sigma dt\label{eq:c2_varham_T2}.
\end{align}
Since the variations are fixed at the boundaries of $[t_0,t_1]$, the first right hand part in \eqref{eq:c2_varham_T2} vanishes. 

The internal strain energy of the soft robot is defined as{}
\begin{equation}
\mathcal{V}_{in} := \frac{1}{2}\int_\mathcal{X} \vec{\xi}^\tr\! \vec{\Lambda} \;d\sigma.
\end{equation}

where $\vec{\Lambda} \in \R^6$ is the vector field representing the internal forces of the stress resultants over the continuum body. These internal force vector field and the strains vector field are related through a material constitutive law. In general concerning soft robotic applications, the use of linear constitutive relations for an isotropic elastic material are not sufficient, since large deformations introduce nonlinear material behavior. However, for the sake of simplicity, we consider the simplest viscoelastic constitutive model - the Kelvin-Voigt model. The Kelvin-Voigt model is a linear elasticity model with a linear viscous contribution that is proportional to the rate of strain $\vec{\xi}$, 
\begin{equation}
\vec{\Lambda} = \mat{K}\vec{\xi} + \mat{\Gamma} \dot{\vec{\xi}}
\end{equation}
where $\mat{K}$ and $\mat{\Gamma}$ are the elasticity and viscosity material tensor, respectively.


% \clearpage
% \clearpage
% \section{Kinematics}
% Suppose the local deformation of the soft robot can be described by a generalized coordinate vector $\boldsymbol{q}(t) \in \R^n$ with $n$ the number of degrees of freedom. Since a soft robot is theoretically a continuum elastic solid with no well-defined joints, the vector $\boldsymbol{q}(t)$ can be chosen arbitrarily to express the geometric deformations of the body. However, as will become apparent during this chapter, proper choice on the joints of the soft robot can significantly reduce modeling complexity. To represent the posture of the soft robot, we introduce a continuously differentiable curve in $\R^3$ space passing through the geometric center of the soft robot. We define the elastic body according to a modified version of the Cosserat rod theory, that is,            
% \begin{equation}
% \boldsymbol{p}: \R^n \times \Rp \mapsto \R^3
% \end{equation}

% For every point on this spatial curve, we introduce an extend Frenet frame given by an orthonormal basis $^\sigma\!\boldsymbol{e} = \{^\sigma \boldsymbol{e}_x,\,^\sigma \boldsymbol{e}_y,\,^\sigma \boldsymbol{e}_z\}$. The rotation matrix associated with this Frenet frame is defined by $\Rot{0} \in SO(3)$ belonging to the special orthogonal group. 

% The orientation matrix is described as
% \begin{equation}
% \Rot{0}(\q,\sigma) = \begin{pmatrix}
% {s_\phi}^2v_a + c_a & -s_\phi c_\phi v_a & c_\phi s_a \\ 
% -c_\phi s_\phi v_a & {c_\phi}^2v_a + c_a & s_\phi s_a \\
% -c_\phi s_a & -{s_\phi}s_a & c_a \\
% \end{pmatrix},
% \label{eq:dyn_rotmat}
% \end{equation}
% with abbreviated notations $c_a = \cos(\sigma \kappa)$, $s_a = \sin(\sigma \kappa)$, $v_a = 1-\cos(\sigma \kappa)$, $c_\phi = \cos(\phi)$, $s_\phi = \sin(\phi)$, and $\phi = \text{atan2}(\kappa_x,\kappa_y)$. 
% \section{Continuous dynamics}
% Introduced by H. Pointcaré in 1988, it was shown that the dynamics in Cosserat beam theory can be directly derived from an extension of continuum media of variational calculus. Contrary to Lagrangian mechanics, it allows the derivation of dynamical systems whose configuration spaces require the structure of Lie groups. Within this formulation, the dynamics of the soft robot can be derived from the Lagrangian density as $\mathcal{L} := \mathcal{T}(\boldsymbol{q},\dot{\boldsymbol{q}},\sigma) - \mathcal{V}(\boldsymbol{q},\sigma)$, where $\mathcal{T}$ and 

% \subsection{Planar case}
% For planar continuum-bodied motion, we postal that $\tilde{\boldsymbol{\Phi}} = \lim_{\kappa_y \to 0} \left[ \boldsymbol{\Phi}(\boldsymbol{q},\sigma) \right]$. Thus, the rotation matrix in \eqref{eq:dyn_rotmat} reduces to
% \begin{equation}
% \Rot{0}(\kappa,\sigma) = \begin{pmatrix}
% \cos(\sigma \kappa) & 0 & \sin(\sigma \kappa) \\ 
% 0 & 1 & 0 \\
% -\sin(\sigma \kappa) & 0 & \cos(\sigma \kappa) \\
% \end{pmatrix}, 
% \end{equation}

% \begin{definition}[Linear rotations]
% There are several ways to express a rotation in 
% In some cases, it might be interesting to consider curvatures that have a dependency on the along the spatial curve $\boldsymbol{p}$, that is, $\kappa(\sigma_i) \neq \kappa(\sigma_j)\; \forall \sigma_i,\sigma_j \in [0,l]$. Therefore, we can write the rotation matrix as a summation of skew-symmetric matrix $\Rot{0}(\boldsymbol{q},\sigma)  =  \sum_{i = 1}^{\infty} \frac{1}{k!} \left[\boldsymbol{S}\left(\sigma \boldsymbol{\theta}\right)\right]^{k} $

% \end{definition}

