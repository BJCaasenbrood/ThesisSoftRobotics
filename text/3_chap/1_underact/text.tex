%!TEX root = C:\Users\s118759\Documents\GitHub\ThesisSoftRobotics\main.tex

% \section{Definitions}
% The general class of the second-order dynamical system can be written as a system of differential equations of the form
% \begin{equation}
% \ddot{q} = f(q,\dot{q},u,t),
% \end{equation}
% where $x = [q,\,\dot{q}]^\tr$ is a state vector composed of the generalized coordinate vector $q \in \R^n$ and their velocities $\dot{q} \in R^n$, and $u \in \R^n$ the control vector. Although for most robotic systems the dynamics are nonlinear in nature, particularly for robots with large degrees-of-freedom, most dynamical models can be written in the control-affine form
% \begin{equation}
% \dot{ \vec{x}} = \vec{f}(\vec{x},t) +  \sum_{i=1}^n \vec{g}_i(\vec{x}) \vec{u}_i(t),
% \label{eq:set_dyn_control}
% \end{equation}
% where $g_i: \R^n \to \R^n$ is a vector-valued function describing the direction of the $i$-th control input, whereas $u_i(t) \in \R$ ,i.e., the amplitude of the control input.

% \begin{definition} [Under-actuated system] A dynamical system described by \eqref{eq:set_dyn_control} is considered under-actuated iff there exists a configuration ${\vec{x}} = [q,\,\dot{q}]^\tr$, such that a set of instantaneous accelerations $\dot{x}$ cannot be realized by the input $u$. Mathematically, it can inferred that a dynamical system is (locally) under-actuated if 
% \begin{equation}
% \exists x \;\; \rank \left( \sum_{i=1}^n \vec{g}_i(\vec{x},t) \right) < \dim(\vec{x}).
% \end{equation}
% \end{definition}

%\newpage
Before addressing the controller synthesis, we briefly introduce the notion of an under-actuated dynamical system.

\begin{definition}[Under-actuated system]
A second-order dynamical system describe by the partial differential equation
\begin{equation}
\ddot{q} = f(q,\dot{q},u,t)
\end{equation}
is considered fully-actuated in a state $(q,\dot{q})$ at time $t$ iff the map $f$ is surjective, i.e, for every $\ddot{q}(t)$ there exists a control input $u(t)$ such that the instantaneous acceleration is realizable. Otherwise, the dynamical system is said to be under-actuated. Regarding under-actuated systems in the control-affine form, that is,
\begin{equation}
\ddot{q} = f_1(q,\dot{q},t) + f_2(q,\dot{q},t)u(t),
\end{equation}
a sufficient condition is $\rank \left(f_2(q,\dot{q},t)\right) < \dim(q)$.
\end{definition}

\noindent In other words, an under-actuated dynamical system cannot steer its states in any arbitrary direction. As a consequence, underactuated systems are generally more difficult to control. By definition, a soft robotic system is an under-actuated system since they theoretically poses infinitely many degrees-of-freedom. Including the distributed control inputs to the Lagrangian model, we write the dynamics for a soft robotic system as follows
\begin{equation}
M(q)\ddot{q} + c(q,\dot{q}) + g(q)= S(q) u,
\end{equation}

\noindent where $M(q)$ is the positive definite mass matrix, $c(q,\dot{q})\in \R^{n}$ is a vector of Coriolis forces, $g(q)\in \R^{n}$ a vector of conservative potential forces, $S(q) \in \R^{n\times m}$ is a (nonlinear) mapping that projects the active control inputs onto the acceleration joint space of ${q}$, and $u(t) \in \R^{m}$ is the lower-dimensional control input. Since the system is under-actuated, it shall be clear that $\dim(u) < \dim(q)$. In literature \cite{Santina2019}, the matrix $S$ is referred to as the synergy matrix whose columns describe actuation patterns of the soft robot's input space. Without loss of generality, let the synergy matrix $S$ be defined by a set of linearly independent column vectors of actuation patterns $s_i: \R^n \to \R^n$,
\begin{align}
S(q) := \left\{\,s_1(q),\,s_2(q),\,...,\,s_m(q)\,\right\}, 
\end{align}
which implies that the matrix has full row-rank $\rank(S) = m$. 
% \begin{equation}
% S = \row \left\{B^\tr \!J(0,\sigma_1,q),\, B^\tr\!J(\sigma_1,\sigma_2,q),\, ...., B^\tr\![J(\sigma_m,q) - J(\sigma_{m-1},q)] \right\}
% \end{equation}

% \noindent where $J(\sigma_i,q) \in \R^{6\times n}$ is a geometric Jacobian matrix of the $i$-th control input related to a particular point $\sigma_{i-1} < \sigma_i < \sigma_{i+1} $ with $i \in [1,m]$ on the continuous body, and $B \in \R^{6\times n_a}$ is a selection matrix for the generalized wrenches in $\R^6$. 
\section{Operational space formulation}
The task of a robotic system is commonly formulated in terms of the end-effector position, velocity, or environmental force. As such, regardless of the dimensionality of the joint space, the control synthesis is typically specified by subpart of the robotic system. Let us introduce a state vector $x \in \R^k$ denoting the task or operational space of the soft robotic system with $\dim(x) < \dim(q)$. Mathematically, the operational space can be described using a continuously differentiable output map $h: \R^n \mapsto \R^k$ such that $x = h(q)$. Differentiation with respect to time gives rise to
\begin{align}
\dot{x} & = J(q) \dot{q}, \label{eq:xJq}\\
\ddot{x} & = \dot{J}(q)\dot{q} + J(q) \ddot{q} \label{eq:dxJdq}, 
\end{align}
where $J(q):= \frac{\p h}{\p {q}}(q)$ is defined as the analytical Jacobian matrix. Following the work of Khatib (1987), we can use the kinematic equalities \eqref{eq:xJq} and \eqref{eq:dxJdq} to derive the dynamics of the underactuated system expressed in the operational space formulation as follows
\begin{equation}
\Lambda_x(q)\ddot{x} + \mu_x(q,\dot{q}) + p_x(q) = F,
\end{equation}
where the reduced components $\Lambda_x$, $\mu_x$, and $p_x$ take the form 
\begin{align}
\Lambda_x(q) & = \left(J^\tr M^{-1} J \right)^{-1}, \\
\mu_x(q,\dot{q}) & = \left(J^\ginv\right)^\tr c(q,\dot{q}) - \Lambda_x(q) \dot{J} \dot{q}, \\
p_x(q) & = \left(J^\ginv\right)^\tr g(q), \\
F & = \left(J^\ginv\right)^\tr S(q) u,
\end{align}
with the matrix $J^\ginv$ denoting the generalized inverse of $J$ that satisfies $J = J J^\ginv J$. If $J^\ginv$ satisfies the four Moore–Penrose conditions, then the generalized inverse is equivalent to Moore-Penrose's inverse of $J$, also known as the pseudo-inverse. Another widely used generalized inverse is a dynamically consistent generalized inverse first proposed by Khatib (1987). This dynamically consistent generalized inverse is given by
\begin{equation}
J^\ginv(q) = M^{-1}\! \, J^\tr\!\, \Lambda_x,
\end{equation}
which gives the lowest instantaneous kinetic energy $\frac{1}{2} \langle \dot{q}, M\dot{q} \rangle$ while minimizing the least squares $\norms{J \dot{q} - \dot{x}}$. To clarify, given a predefined trajectory in $\dot{x}$, there is exists multiple solutions to $(q,\dot{q})$ that satisfy the task-velocity $\dot{x} = J(q)\dot{q}$, as illustrated by $\dot{q} = J^\ginv \dot{x} + (I - J^\ginv J)b$ with $b$ arbitrary configuration velocity in $\R^n$. This generalized inverse satisfies the first three Moore–Penrose conditions with an additional condition $M J^\ginv J = \bk{M J^\ginv J}^\tr$

\section{Geometric Jacobian of $\SE{3}$}
The geometric Jacobian can be computed through finding the solutions of the configuration space $g$ and an unknown integrand $\Xi$ over the domain $[0,\sigma]$:
\begin{equation}
\dfrac{\p }{\p \sigma} \begin{pmatrix} 
\;g\;\\ \;\Xi\;
\end{pmatrix} = 
\begin{pmatrix} 
g \hat{\xi} \\
\Ad_{g} \Phi(\sigma)
\end{pmatrix},
\label{eq:jacobi_solve}
\end{equation}
with initial conditions $g(0) = g_0$. Recall that the strain field $\hat{\xi} \in \se{3}$ is approximated using a finite set of shape functions $\hat{\xi} = (\Phi(\sigma)q + \xi_0)^\wedge$. Then, given the solution to the ordinary differential equation \eqref{eq:jacobi_solve}, the geometric Jacobian associated with a point $\sigma$ can be evaluated using $J(\sigma,q) = \Ad_g^{-1} \Xi(\sigma,q)$.
Taking the partial derivate with respect to time of the Jacobian matrix, we may express the change in the geometric Jacobian as
\begin{align}
\dot{J} & = \dot{\Ad}_g^{-1} \Xi + {\Ad}_g^{-1} \dot{\Xi}, \notag \\
& = -ad_{\eta} {\Ad}_g^{-1} \Xi + {\Ad}_g^{-1} \int_0^\sigma ad_{\eta(s)} {\Ad}_{g(s)} \Phi(s) \; ds \label{eq:Jdot}. 
\end{align}
Combining the formulations in \eqref{eq:jacobi_solve} and \eqref{eq:Jdot}, the geometric Jacobian and its time derivative can be computed through finding the solutions of the following ODE
\begin{equation}
\dfrac{\p }{\p \sigma} \begin{pmatrix} 
\;g\;\\ \;\eta\; \\ \;\Xi\; \\ \;\dot{\Xi}\;
\end{pmatrix} = 
\begin{pmatrix} 
g \hat{\xi} \\
-\ad_{\xi} \eta + \dot{\xi} \\
\Ad_{g} \Phi(\sigma) \\
ad_{\eta} {\Ad}_{g} \Phi(\sigma)
\end{pmatrix},
\label{eq:full_jacobi_solve}
\end{equation}
given the initial conditions $g(0) = g_0$ and $\eta(0) = \eta_0$. 